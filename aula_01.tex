\documentclass{article}
\usepackage{indentfirst}
\author{Arthur Von Groll dos Santos}
\title{Aula 01}
\begin{document}
\maketitle

\begin{itemize}
    \item Apresentação de cada um dos alunos
    \item Introdução à História do Brasil
\end{itemize}

\section*{\centering História do Brasil}

O Brasil foi `descoberto' (invadido, pois já havia habitantes lá)
em 22 de abril de 1492 por Pedro Álvares Cabral. No início, o Brasil não era de interesse dos portugueses,
apenas eram utilizados os ventos e correntes marítimas próximos do continente para ir
mais rapidamente à Índia. Porém, como os franceses estavam se `apossando' dos territórios brasileiros,
os portugueses decidiram colocar postos para garantir que o território não seria tomado.

Os portugueses, então, tiveram seu primeiro contato com os povos nativos. Entretanto, devido à diferença
exorbitante entre a cultura européia e a dos indígenas brasileiros, os portugueses acharam que deveriam catequizar
e impor sua cultura nos indígenas.

Algumas diferenças culturais importantes entre os indígenas e os europeus são:

\begin{itemize}
    \item \textbf{Religião}: Os europeus eram católicos, enquanto as tribos indígenas
        tinham diferentes crenças e eram politeístas;
    \item \textbf{Estilo de vida}: Os indígenas trabalhavam conforme o necessário, realizando pausas, enquanto
        os portugueses trabalhavam com um horário rígido.
\end{itemize}

\end{document}